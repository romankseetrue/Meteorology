\documentclass{article}

\usepackage[utf8]{inputenc}
\usepackage{amsthm}

\title{Exploratory Data Analysis}
\author{Roman Kushnirenko}

\begin{document}

	\maketitle

	\section{Introduction}

Exploratory Data Analysis (EDA) uses visualization and transformation to explore your data in a systematic way. EDA is an iterative cycle. You:

\begin{enumerate}
  \item Generate questions about your data.
  \item Search for answers by visualizing, transforming, and modeling your data.
  \item Use what you learn to refine your questions and/or generate new questions.
\end{enumerate}

EDA is not a formal process with a strict set of rules. More than anything, EDA is a state of mind. During the initial phases of EDA you should feel free to investigate every idea that occurs to you. Some of these ideas will pan out, and some will be dead ends. As your exploration continues, you will hone in on a few particularly productive areas that you’ll eventually write up and communicate to others.

EDA is an important part of any data analysis, even if the questions are handed to you on a platter, because you always need to investigate the quality of your data. Data cleaning is just one application of EDA: you ask questions about whether or not your data meets your expectations. To do data cleaning, you’ll need to deploy all the tools of EDA: visualization, transformation, and modeling.

	\section{Questions}

Your goal during EDA is to develop an understanding of your data. The easiest way to do this is to use questions as tools to guide your investigation. When you ask a question, the question focuses your attention on a specific part of your dataset and helps you decide which graphs, models, or transformations to make.

EDA is fundamentally a creative process. And like most creative processes, the key to asking \textit{quality} questions is to generate a large \textit{quantity} of questions. It is difficult to ask revealing questions at the start of your analysis because you do not know what insights are contained in your dataset. On the other hand, each new question that you ask will expose you to a new aspect of your data and
increase your chance of making a discovery. You can quickly drill down into the most interesting parts of your data \(-\) and develop a set of thought-provoking questions \(-\) if you follow up each question with a new question based on what you find.

There is no rule about which questions you should ask to guide your research. However, two types of questions will always be useful for making discoveries within your data. You can loosely word these questions as:

\begin{enumerate}
  \item What type of variation occurs within my variables?
  \item What type of covariation occurs between my variables?
\end{enumerate}

	\section{Variation}

\textit{Variation} is the tendency of the values of a variable to change from measurement to measurement. Every variable has its own pattern of variation, which can reveal interesting information. The best way to understand that pattern is to visualize the distribution of variables’ values.

\textbf{Visualizing Distributions.} How you visualize the distribution of a variable will depend on whether the variable is categorical or continuous.

A variable is \textit{categorical} if it can only take one of a small set of values. To examine the distribution of a categorical variable, use a bar chart. The height of the bars displays how many observations occurred with each \(x\) value.

A variable is \textit{continuous} if it can take any of an infinite set of ordered values. To examine the distribution of a continuous variable, use a histogram. A histogram divides the \(x\)-axis into equally spaced bins and then uses the height of each bar to display the number of observations that fall in each bin. You should always explore a variety of binwidths when working with histograms, as different binwidths can reveal different patterns.

\textbf{Typical Values.} In both bar charts and histograms, tall bars show the common values of a variable, and shorter bars show less-common values. Places that do not have bars reveal values that were not seen in your data. To turn this information into useful questions, look for anything unexpected:

\begin{itemize}
	\item Which values are the most common? Why?
	\item Which values are rare? Why? Does that match your expectations?
	\item Can you see any unusual patterns? What might explain them?
\end{itemize}

In general, clusters of similar values suggest that subgroups exist in your data. To understand the subgroups, ask:

\begin{itemize}
	\item How are the observations within each cluster similar to each other?
	\item How are the observations in separate clusters different from each other?
	\item How can you explain or describe the clusters?
	\item Why might the appearance of clusters be misleading?
\end{itemize}

\textbf{Unusual Values.} Outliers are observations that are unusual; data points that don’t seem to fit the pattern. Sometimes outliers are data entry errors; other times outliers suggest important new science.

It’s good practice to repeat your analysis with and without the outliers. If they have minimal effect on the results, and you can’t figure out why they’re there, it’s reasonable to replace them with missing values and move on. However, if they have a substantial effect on your results, you shouldn’t drop them without justification. You’ll need to figure out what caused them and disclose that you removed them in your write-up.

\textbf{Missing Values.} If you’ve encountered unusual values in your dataset, and simply want to move on to the rest of your analysis, you have two options:

\begin{itemize}
\item \textit{Drop the entire row with the strange values.} I don’t recommend this option as just because one measurement is invalid, doesn’t mean all the measurements are. Additionally, if you have low-quality data, by time that you’ve applied this approach to every variable you might find that you don’t have any data left!
\item Instead, I recommend \textit{replacing the unusual values with missing values.}
\end{itemize}

\end{document}\grid
