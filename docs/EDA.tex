\documentclass{article}

\usepackage[utf8]{inputenc}
\usepackage{amsthm}

\title{Exploratory Data Analysis}
\author{Roman Kushnirenko}

\begin{document}

	\maketitle

	\section{Introduction}

Exploratory Data Analysis (EDA) uses visualization and transformation to explore your data in a systematic way. EDA is an iterative cycle. You:

\begin{enumerate}
  \item Generate questions about your data.
  \item Search for answers by visualizing, transforming, and modeling your data.
  \item Use what you learn to refine your questions and/or generate new questions.
\end{enumerate}

EDA is not a formal process with a strict set of rules. More than anything, EDA is a state of mind. During the initial phases of EDA you should feel free to investigate every idea that occurs to you. Some of these ideas will pan out, and some will be dead ends. As your exploration continues, you will hone in on a few particularly productive areas that you’ll eventually write up and communicate to others.

EDA is an important part of any data analysis, even if the questions are handed to you on a platter, because you always need to investigate the quality of your data. Data cleaning is just one application of EDA: you ask questions about whether or not your data meets your expectations. To do data cleaning, you’ll need to deploy all the tools of EDA: visualization, transformation, and modeling.

	\section{Questions}

Your goal during EDA is to develop an understanding of your data. The easiest way to do this is to use questions as tools to guide your investigation. When you ask a question, the question focuses your attention on a specific part of your dataset and helps you decide which graphs, models, or transformations to make.

EDA is fundamentally a creative process. And like most creative processes, the key to asking \textit{quality} questions is to generate a large \textit{quantity} of questions. It is difficult to ask revealing questions at the start of your analysis because you do not know what insights are contained in your dataset. On the other hand, each new question that you ask will expose you to a new aspect of your data and
increase your chance of making a discovery. You can quickly drill down into the most interesting parts of your data \(-\) and develop a set of thought-provoking questions \(-\) if you follow up each question with a new question based on what you find.

There is no rule about which questions you should ask to guide your research. However, two types of questions will always be useful for making discoveries within your data. You can loosely word these questions as:

\begin{enumerate}
  \item What type of variation occurs within my variables?
  \item What type of covariation occurs between my variables?
\end{enumerate}

\end{document}\grid
